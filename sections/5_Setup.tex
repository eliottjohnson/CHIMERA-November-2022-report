
% The silicon diode used to asses the characteristics of the CHIMERA heavy-ion beams was a fully depleted passivated implanted planar silicon (PIPS) detector manufactured by Canberra, model \hbox{FD 50-14-300 RM} (denoted as "\textit{Vanessa I}" diode for simplicity). This diode is \SI{308}{\micro\meter} thick and has an active surface of \SI{0.5}{\centi\meter\squared}. This corresponds to the silicon surface in the circular opening of the metal case of the diode with a radius of \SI{111}{\milli\meter}, as it can be seen in Figure~.... A part of the silicon die is also present under the metal case, for a better mechanical stability. The total size of the silicon can reach a radius of up to \SI{13}{\milli\meter}. However, the exact size of the silicon die in this detector was not revealed by the manufacturer and hence, it is not clear how large is the fraction of silicon under the casing also contributing to the particle detection. This diode has an entrance window of $\leq$ \SI{50}{\nano\meter}.

% % To be checked:
% During the test, the diode was wrapped up with two layers of standard commercial \SI{14}{\micro\meter}-thick aluminium foil (i.e.~{\SI{28}{\micro\meter}} of aluminium in front of the diode) to shield it from light and electromagnetic noise.


% The diode was attached to a Plexiglass support plate using a 3D-printed holder, shown in Fig... and the plexiglass plate was attached on the large standard MOMNTRAC frame placed in the most downstream of the three MONTRAC slots.



% % Preapmplifier:
% The diode was used together with a \SI{20}{\deci\bel} current \gls{pa} from Cividec, model C1HV0089. The certified gain of this \gls{pa} is \SI{21.9}{\deci\bel}. The intensity of the beam was high enough, hence no amplification was needed for this test. The \gls{pa} was used for purely practical reasons: to split the detector bias and signal into two separate lines, while they are exiting the diode through a single connector. A \SI{6}{\deci\bel} attenuator was used to partially compensate for this amplification.

% The diode connected to the \gls{pa} with a \SI{20}{\centi\meter}-long cable were both placed on the MONTRAC movable table, as it is shown in Figure X. The \gls{pa} was connected to the rest of the test setup placed outside of the radiation room with three \SI{20}{\meter}-long BNC cables (and the necessary adapters) running through the maze. One BNC cable was used to provide the recommended reverse bias voltage of $+$\SI{200}{\volt} to the detector, supplied by a \gls{smu} Keithley 2410. The second cable was used to power the \gls{pa} with \SI{12}{\volt} DC voltage supplied by another \gls{smu} Keithley 2410 and the third cable was used to transmit the detector signal to the CEAN DT5751 digitizer before being acquired and saved by the acquisition laptop.